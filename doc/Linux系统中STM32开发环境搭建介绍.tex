\documentclass{article}

\usepackage[BoldFont,SlantFont]{xeCJK}
\setmainfont{Courier New}
\setCJKmainfont{WenQuanYi Micro Hei}
\parindent 2em

\usepackage[superscript]{cite}
\usepackage{graphicx}

\renewcommand{\refname}{参考文献}

\begin{document}
\title{Linux 系统中 STM32 开发环境搭建介绍}
\author{Mars Wu (武晓宇)}
\date{2014/12/04}
\maketitle

\section{简介}
\label{sec:简介}

在 Linux 下搭建 STM32 开发环境,就是配置编译工具链 、工程管理工具以及调试烧录工具的过程 。本文简单说明了利用 GCC 交叉编译工具链 + eclipse + GDB 作为开发 STM32 环境的配置过程 。
\section{安装编译工具链}
\label{sec:编译工具链}

"GNU Tools for ARM Embedded Processors"\cite{gcc-arm-embedded} 是适用于 ARM Corex-M 以及 Cortex-R 处理器的一套工具链 。使用这套工具链就可以编译链接并生成在 STM32 处理器上运行的程序 。它可在其项目主页提供的下载链接地址\cite{gcc-arm-embedded+download}进行下载安装 ,也可以在一些 Linux 发行版的软件管理中找到并安装 。安装好工具链后 ,可用  "arm-none-eabi-*" 调用对应工具命令 。

\section{安装工程管理工具}
\label{sec:工程管理工具}

Linux 系统中,可以使用 Make 工具及 Makefile 进行项目管理,但由于我对 Makefile 不熟悉 ,而且可能需要自己写启动初始化代码,甚至可能要写链接脚本,所以暂时使用 Eclipse CDT 作为工程管理工具。

\subsection{安装配置}
\label{sec:安装配置}

Eclipse CDT 是为 C/C++ 开发者提供的 IDE ,"GNU ARM Eclipse Plug-ins"\cite{CDT-ARM} 是在 CDT 基础上为 ARM 处理器提供工程模板的扩展包 。

\subsubsection{安装 Eclipse}
\label{sec:安装eclipse}

首先在 Eclipse 官网上下载 "luna-SR1" 版本的压缩包\cite{eclipse-download} ,解压并将解压目录加入 PATH 环境变量 ,即可在终端中输入并运行 eclipse 。

\subsubsection{安装 ARM 扩展包}
\label{sec:arm扩展包}

接着打开 eclipse ,在菜单栏中找到并点击 "Help->Install New Software" ,添加站点 \\
http://gnuarmeclipse.sourceforge.net/updates \\
从中安装 "GNU ARM C/C++ Cross Development Tools" ,等待安装完毕即可 。

\begin{center}
  \includegraphics[width=5cm]{install1}
\end{center}

如果安装过程中出现错误 ,找不到 CDT 工具之类的 ,可以添加站点 \\
CDT - http://download.eclipse.org/tools/cdt/releases/8.5 \\安装好 "CDT Main Features" "CDT Optional Features" ,之后重新安装 "GNU ARM C/C++ Cross Development Tools" 。


\subsection{创建工程过程}
\label{sec:创建工程过程}

安装好 CDT 及 ARM 扩展包之后 ,进入 Eclipse ,单击 "File->New->C Project" ,在左侧的 "Executable" 中可看到 STM32 Project ,根据芯片型号选择对应工程 ,单击 "Next" 。
\begin{center}
  \includegraphics[width=5cm]{step1}
\end{center}

接着选择芯片种类 ,填写 Flash 大小等信息 ,选择工程内容及编译选项 ,继续单击 "Next" 。
\begin{center}
  \includegraphics[width=5cm]{step2}
\end{center}


保持默认配置到最后一步,选择之前安装的 "GNU Tools for ARM Embedded Processors" 工具链的地址 ,点击 "Finish",工程创建完毕 。其中 ,STM32 的 CMSIS 库已经被自动添加到工程目录,接下来只需填写自己的应用程序代码 ,编译生成可执行文件 。
\begin{center}
  \includegraphics[width=5cm]{step3}
\end{center}
\begin{center}
  \includegraphics[width=5cm]{step4}
\end{center}


注意 :工程的 Makefile 文档不可修改 ,默认的编译选项可能和交叉编译工具有冲突错误 ,建议安装 GCC 交叉工具链时下载官方最新的版本 。

\section{安装调试烧录工具}
\label{sec:安装调试烧录工具}

调试烧录工具使用的是 ST 的 stlinkv2 ,没有官方的 Linux 系统驱动 。Github 中有 Linux 操作 stlinkv1 及 stlinkv2 的软件工具 stlink ,链接为 \\
https://github.com/texane/stlink \\
安装该软件前,需要安装 GDB 、GDB Server 、libusb-1.0 、pkg-config 这 4 个软件。下载 stlink 后按照下载源码包中的 README 文档说明进行编译安装 。安装好后 ,就可以在终端中使用 "st-util" 命令 ,打开 GDB Server 并连接到 STM32 设备 。
\begin{center}
  \includegraphics[width=5cm]{step5}
\end{center}

连接到了 STM32 设备 ,接下来就用 "arm-none-eabi-gdb" 命令打开编译好的可执行文件 ,
用 "tar extended-remote: 4242" 命令连接到 GDB Server ,用 “load” 命令进行烧录 。烧录完成后 ,就可以调试运行了 。
\begin{center}
  \includegraphics[width=5cm]{step6}
\end{center}

使用 GDB 调试时 ,以下两点需要注意 :
\begin{itemize}
\item 可以设置数据断点 ,但是执行速度非常慢 ,甚至 stm32 死机 ,建议设置程序断点即可 。
\end{itemize}
\begin{itemize}
\item 有时 GDB 无法读取或者写入某些内存地址的值 ,比如 stm32 的 FSMC 寄存器区域 ,调试时程序运行到相应语句时会导致 stm32 PC 指针指向异常 ,还会导致之后的程序烧录异常 。建议在烧录程序后 ,退出 GDB 调试环境 ,直接在 stm32 中上电运行程序 ,采用串口打印信息的方式进行代码调试 。
\end{itemize}

\section{其它}
\label{sec:其它}

\subsection*{串口调试工具}
\label{sec:串口调试工具}

Linux 下的串口调试工具有 Minicom 、ckert 等,在此建议使用 cutecom 软件 ,图形化配置收发。
\begin{center}
  \includegraphics[width=5cm]{step7}
\end{center}
\begin{thebibliography}{}
\bibitem{gcc-arm-embedded}
  https://launchpad.net/gcc-arm-embedded
\bibitem{gcc-arm-embedded+download}
  https://launchpad.net/gcc-arm-embedded/+download
\bibitem{CDT-ARM}
  http://sourceforge.net/projects/gnuarmeclipse/files/Eclipse/updates/
\bibitem{eclipse-download}
  http://www.eclipse.org/downloads/packages/eclipse-ide-cc-developers/lunasr1
\end{thebibliography}
\end{document}
